\section{Background}
\label{s:background}
%
In this section, we define as our target language an extension of the
simply-typed $\lambda$-calculus (STLC) (\autoref{sec:stlc}).
%
We then define partial equivalence of STLC programs
(\autoref{sec:peq}).

\subsection{The Simply-Typed $\lambda$-calculus}
\label{sec:stlc}
\begin{figure}[t]
  \begin{floatrow}[2]
    \ffigbox[.48\textwidth] %
    { \caption{The space of program expressions, $\expr$.} %
      \label{fig:exprs} }
    {  \begin{align*}
        % expression is a constant,
        e \rightarrow &\ c \\
        % or a variable,
        | &\ x \\
        % or a binary operation,
        | &\ e \oplus e \\
        % or an if-then-else, 
        | &\ \ite{ e }{ e }{ e } \\
        % or an abstraction,
        | &\ \lambda x.\ e \\
        % or an application,
        | &\ e\ e \\
        % or a fixpoint,
        | &\ \fix{ f }{ e }
      \end{align*} }
    %
    \ffigbox[.48\textwidth] %
    {\caption{The space of types, $\types$.
        %
        Metavariable $\tau$ ranges over $\types$.} %
      \label{fig:types} }
    %
    { \begin{align*} %
        \tau \rightarrow &\ \base\ %
        |\ \base \to \tau \\
        \base \rightarrow &\ \bools\ %
        |\ \ints
      \end{align*} }
  \end{floatrow}
\end{figure}
%
\BH{lift scope of metavariables}

% define STLC syntax:
\paragraph{Syntax} \autoref{fig:exprs} defines the space of
simply-typed $\lambda$-calculus expressions, $\expr$.
%
For the remainder of this paper, metavariable $e$ ranges over
$\expr$, %
$c$ ranges over constants, %
$\oplus$ ranges over binary operators, and $x$ and $f$ range over
program variables.
%
An expression is a Boolean or integer constant, %
a variable (drawn from space of variables $\vars$), %
an application of an fixed operator to a pair of expressions, %
an \cc{if}-\cc{then}-\cc{else} expression, %
a $\lambda$-abstraction, %
an application, or %
a fixpoint expression.
%
For each $e \in \expr$, the \emph{free-variables} of $e$ are denoted
$\freevars(e)$.
%
A STLC \emph{program} is a closed expression---i.e., an expression $e$
such that $\freevars(e) = \emptyset$.
%
The space of STLC programs is denoted $\stlc$.
%
For $P \in \stlc$, the set of all subexpressions of $P$ is denoted
$\subexps(P)$.

% define semantics:
\paragraph{Semantics}
%
\begin{figure}
  \centering
  \begin{gather*}
  % evaluate const:
  \inference[ CONST ]{ }{ \Gamma \judges c \evalsto c } 
  % evaluate variable:
  \inference[ VAR ]{ }{ \Gamma \judges x \evalsto \Gamma(x) }
  % evaluate binary operations:
  \inference[ OP ]{ %
    \Gamma \judges e_0 \evalsto v_0 & %
    \Gamma \judges e_1 \evalsto v_1 } %
  { \Gamma \judges e_0 \oplus e_1 \evalsto v_0 \oplus v_1 } \\
  \\
  % evaluate if-then-else by then branch
  \inference[ ITE-T ]{ %
    \Gamma \judges e_0 \evalsto \true & %
    \Gamma \judges e_1 \evalsto v_1 } %
  { \Gamma \judges \ite{ e_0 }{ e_1 }{ e_2 } \evalsto v_1 }
  % evaluate if-then-else by else branch
  \inference[ ITE-F ]{ %
    \Gamma \judges e_0 \evalsto \false & %
    \Gamma \judges e_2 \evalsto v_2 } %
  { \Gamma \judges \ite{ e_0 }{ e_1 }{ e_2 } \evalsto v_2 } \\
  \\
  % evaluate an abstraction:
  \inference[ ABS ]{ } %
  { \Gamma \judges \lambda x.\ e \evalsto \lambda x.\ e }
  % evaluate a fix:
  \inference[ FIX ]{ } %
  { \Gamma \judges \fix{ f }{ e } \evalsto \fix{ f }{ e } } \\
  \\
  % apply a lambda:
  \inference[ APP-$\lambda$ ]{ %
    \Gamma \judges e_0 \evalsto \lambda x.\ e_0' & %
    \Gamma \judges e_1 \evalsto v_1 & %
    \upd{\Gamma}{x}{v_1} \judges e_0' \evalsto v }
  { \Gamma \judges e_0\ e_1 \evalsto v } \\
  \\
  % apply a fixpoint:
  \inference[ APP-FIX ]{ %
    \Gamma \judges e_0 \evalsto \fix{ f }{ e_0' } & %
    \upd{ \Gamma }{ f }{ \fix{ f }{ e_0' } } \judges %
    e_0'\ e_1 \evalsto v }
  { \Gamma \judges e_0\ e_1 \evalsto v } 
  \end{gather*}
  \caption{The natural semantics of $\stlc$.}
  \label{fig:nat-sem}
\end{figure}
%
A value is a Boolean, an integer, a $\lambda$ abstraction, or a
fixpoint abstraction;
%
the space of values is denoted $\values$.
%
The semantics of $\stlc$ is defined by a natural semantics that
relates expressions to values, given in \autoref{fig:nat-sem}.
%
For $e \in \expr$ and $v \in \values$, the fact that $e$
\emph{evaluates} to $v$ is denoted $e \evalsto v$.

\BH{define space of derivations}

\paragraph{Typing}
%
Each expression has a type that describe the spaces of values that it
takes as input and the spaces of values to which it evaluates.
%
The space of types is given in \autoref{fig:types}.
%
A type is either a base type or a function from a base type to a type.
%
The definition of types in \autoref{fig:types} thus does not include
higher-order expressions.
%
We will relax this restriction in future work.
%
\BH{It's probably reasonable to relax this assumption before a first
  paper}

% introduce has-type symbol:
For expression $e \in \expr$ and type $\tau \in \types$, the fact that
$e$ has type $\tau$ is denoted $e \hastype \tau$.
%
The definition of the has-type relation is standard.

% standard defns for logic:
\paragraph{Logic}
%
\sys uses formal logic to model the semantics of programs.
%
The quantifier-free fragment of the theory of linear arithmetic is
denoted $\lia$.
%
For each space of logical variables $X$, the space of $\lia$ formulas
over $X$ is denoted $\formulas{X}$.
%
For each formula $\varphi \in \formulas{X}$, the set of variables that
occur in $\varphi$ (i.e., the \emph{vocabulary} of $\varphi$) is
denoted $\vocab(\varphi)$.
% define models, satisfaction, entailment
A \lia \emph{model} $m$ over $X$ is an assignment from each variable
in $X$ to an integer.
%
The fact that model $m$ \emph{satisfies} a formula $\varphi$ is
denoted $m \sats \varphi$.
%
For formulas $\varphi_0, \ldots, \varphi_n, \varphi \in \formulas{X}$,
the fact that $\varphi_0, \ldots, \varphi_n$ \emph{entail} $\varphi$
is denoted $\varphi_0, \ldots, \varphi_n \entails \varphi$.

% Define replacement and substitution:
For all vectors of variables $X = [ x_0, \ldots, x_n ]$ and $Y = [
y_0, \ldots, y_n ]$, the \lia formula constraining the equality of
each element in $X$ with its corresponding element in $Y$, i.e., the
formula $\bigland_{0 \leq i \leq n} x_i = y_i$, is denoted $X = Y$.
%
The repeated replacement of variables $\varphi[ y_0 / x_0 \ldots y_{n}
/ x_{n} ]$ is denoted $\replace{\varphi}{Y}{X}$.
%
For each formula $\varphi$ defined over free variables $X$,
$\replace{\varphi}{X}{Y}$ is denoted alternatively as $\varphi[Y]$.
%
The number of free variables in formula $\varphi$ is denoted
$\degreeof(\varphi)$.

% Introduce decision procedure:
Although determining the satisfiability of a \lia formula is
NP-complete in general, decision procedures for \lia have been
proposed that often determine the satisfiability of formulas that
arise from practical verification problems
efficiently~\cite{demoura08}.
%
\sys assumes access to a decision procedure for \lia, denoted \issat.

% introduce interpolants over formula pairs:
An interpolant of mutually inconsistent formulas $\varphi_0$ and
$\varphi_1$ is a \lia formula that explains their inconsistency using
their common vocabulary.
%
\begin{defn}
  \label{defn:itps}
  % define:
  For spaces of logical variables $X$ and $Y$,
  %
  $\varphi_0 \in \formulas{X}$ and
  %
  $\varphi_1 \in \formulas{Y}$,
  % 
  if $I \in \formulas{X \intersection Y}$ is such that 
  %
  \textbf{(1)} $\varphi_0 \entails I$ and %
  \textbf{(2)} $I, \varphi_1 \entails \false$, then $I$ is an
  \emph{interpolant} of $\varphi_0$ and $\varphi_1$.
\end{defn}
%
Previous work has introduced interpolating theorem provers that
synthesize interpolants of pairs of mutually-unsatisfiable formulas in
extensions theories used to model program semantics and
specifications~\cite{mcmillan04}.
%
To present \sys, we assume access to a procedure \solveitp that, given
mutually unsatisfiable \lia formulas $\varphi_0, \varphi_1$, returns
an interpolant of $\varphi_0$ and $\varphi_1$.

\BH{replace with background from shara paper}

\paragraph{Symbolic encoding of semantics}
%
The semantics of $\stlc$ is represented symbolically using \lia
formulas.
%
\BH{hoist defn of disjoint copies of variables for expressions}
%
In particular, for each $e \in \expr$ with subexpressions $e_0,
\ldots, e_n \in \expr$, there is some $\varphi \in \formulas{\vars[e],
  \vars[e_0], \ldots, \vars[e_n]}$ such that

\BH{replace with new problem defn: just checking if a program has a
  type}

%%% Local Variables: 
%%% mode: latex
%%% TeX-master: "p"
%%% End: 
