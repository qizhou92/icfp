% give the real meat of the thing
\section{Technical Approach}
%
In this section, we describe our technical approach in detail.
%
In \autoref{sec:rel-invs}, we define the class of proof
structures synthesized by \sys.
%
In \autoref{sec:verifier}, we present the verification algorithm used
by \sys to synthesize such structures.

\subsection{A class of relational invariants as proofs}
\label{sec:rel-invs}
\BH{define class of relational invariants. Proof is a set of
  subexpressions to a symbolic invariant}

\subsection{An automatic equivalence prover}
\label{sec:verifier}
%
\BH{define equivalence verifier. Maintains relational invariants over
  sets of derivations.}

\BH{derivation refutation: define relational invariants over sets of
  subderivations that form a frontier in the pair of
  derivations. Start from the set of all leaves. Get the relational
  invariant for each set by taking disjunction over its predecessors
  along every step, and then some context formula that will be
  non-trivial to define.}

\BH{after solving for relational invariants over sets of derivations,
  some derivations may occur more than once in a pair of
  derivations. The invariants for these get folded together using
  conjunction. This normalizes the result into a relational invariant
  per set of derivation}

\BH{sets of derivations play the role of sets of paths in path-based
  analysis. Entailment is checked over sets derivations that derive
  the same set of expressions. Search for a sufficient set can branch:
  we just need to find an inductive set along some expression.}

%%% Local Variables: 
%%% mode: latex
%%% TeX-master: "p"
%%% End: 
