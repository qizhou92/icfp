% give the real meat of the thing
\section{Technical Approach}
%
In this section, we describe our technical approach in detail.
%
In \autoref{sec:rel-invs}, we define the class of proof
structures synthesized by \sys.
%
In \autoref{sec:verifier}, we present the verification algorithm used
by \sys to synthesize such structures.

\subsection{Relational invariants as equivalence proofs}
\label{sec:rel-invs}
%
\sys, given programs $P_0$ and $P_1$, attempts to synthesize a proof
that $P_0$ is equivalent to $P_1$ that is represented as set of
subexpressions of $P_0$ and $P_1$, each mapped to relational
invariants that relate all evaluations of the subexpressions.
%
For the rest of this section, let $P_0, P_1 \in \stlc$ be two fixed,
arbitrary $\stlc$ programs, and let $E_{0, 1} = \subexps(P_0) \union
\subexps(P_1)$.

% define symbolic relations:
\begin{defn}
  \label{defn:sym-rels}
  Let $R: \pset(E) \to \formulas{ E_{0, 1} }$ be such that %
  for each $E \subseteq E_{0, 1}$, %
  it holds that $R(E) \in \formulas{ \symvars{E} }$.
  %
  Then $R$ are \emph{symbolic relations}.
\end{defn}
%
The space of symbolic relations is denoted $\symrels$.

% relational invariants
% define relational invariants
Relational invariants of $P_0$ and $P_1$ are a map from sets of
subexpressions of $P_0$ and $P_1$, with each set $S$ mapped to a
formula that relates the evaluations of all expressions in $S$.
%
Relational invariants soundly describe all steps of evaluation that
can be taken by $P_0$ and $P_1$.
\begin{defn}
  \label{defn:rel-invs}
  For $P_0, P_1 \in \stlc$, let $R \in \symrels$ be such that %
  % condition: relation of empty set is entailed by true:
  \textbf{(1)} $\true \entails R(\emptyset)$;
  % condition: steps are valid:
  \textbf{(2)} for each $E \not= \emptyset \in \domain(R)$, %
  there is some $e \in E$ such that either %
  % subcase: P0 takes a step: 
  \textbf{(a)} $e \in \subexps(P_0)$ and for each $r \in
  \evalrules$ that decomposes $e$ to $e_0, \ldots, e_n \in
  \subexps(P_0)$ in $P_0$,
  %
  there are some $\mathcal{E} \subseteq \pset(E_{0, 1})$ with $\elts{
    e_0, \ldots, e_n } = \bigunion \mathcal{E}$ such that
  % 
  \[ \setformer{ R(E') }{ E' \in \mathcal{E} }, %
  \subs{ \symstep{P_0}{ r } }{ %
    \vars[ e_0 ], \ldots, \vars[ e_n ], \vars[ e ] } %
  \entails R(E)
  \]
  or
  % symmetric condition for P1
  \textbf{(b)} $e \in \subexps(P_1)$ for each evaluation rule $r$ that
  decomposes $e$
  %
  to $e_0, \ldots, e_n \in \subexps(P_1)$ in $P_1$,
  %
  there are some $\mathcal{E} \subseteq \pset(E_{0, 1})$ with $\elts{
    e_0, \ldots, e_n } = \bigunion \mathcal{E}$ such that
  % 
  \[ \setformer{ R(E') }{ E' \in \mathcal{E} }, %
  \subs{ \symstep{P_1}{ r } }{ %
    \vars[ e_0 ], \ldots, \vars[ e_n ], \vars[ e ] } %
  \entails R(E)
  \]
  %
  Then $R$ are \emph{relational invariants} of $P_0$ and $P_1$.
\end{defn}
%
The space of relational invariants is denoted $\relinvs$.

% strengthen relational invariants to evidence of equivalence:
If relational invariants $R$ are such that the invariant for $\elts{
  P_0, P_1 }$, combined with the assumption that $P_0$ and $P_1$ are
given equal values, entails that $P_0$ and $P_1$ evaluate to equal
values, then $R$ is evidence of the partial equivalence of $P_0$ and
$P_1$.
%
\begin{defn}
  \label{defn:eq-pf}
  Let $R \in \relinvs$ be such that $R(P_0, P_1), \alpha^0_0 =
  \alpha^1_0 \entails \nu^0 = \nu^1$.
  %
  Then $R$ is \emph{evidence} of $P_0 \equiv P_1$.
\end{defn}

% example of relational invariants:
\begin{ex}
  \label{ex:rel-invs}
  \BH{give example relational invariants}
\end{ex}

% evidence is a valid proof of partial equivalence:
If there is evidence of the equivalence of $P_0$ and $P_1$, then $P_0
\equiv P_1$.
%
\begin{lemma}
  \label{lemma:valid-evidence}
  If there is some $R \in \relinvs$ such that $R$ is evidence of $P_0
  \equiv P_1$, then $P_0 \equiv P_1$.
\end{lemma}

\subsection{An automatic equivalence verifier}
\label{sec:verifier}
%
\BH{define equivalence verifier. Maintains relational invariants over
  sets of derivations.}

\BH{derivation refutation: define relational invariants over sets of
  subderivations that form a frontier in the pair of
  derivations. Start from the set of all leaves. Get the relational
  invariant for each set by taking disjunction over its predecessors
  along every step, and then some context formula that will be
  non-trivial to define.}

\BH{after solving for relational invariants over sets of derivations,
  some derivations may occur more than once in a pair of
  derivations. The invariants for these get folded together using
  conjunction. This normalizes the result into a relational invariant
  per set of derivation}

\BH{sets of derivations play the role of sets of paths in path-based
  analysis. Entailment is checked over sets derivations that derive
  the same set of expressions. Search for a sufficient set can branch:
  we just need to find an inductive set along some expression.}

\BH{sets of derivation nodes that we're interested in are maximal
  anti-chains of nodes under the dependence relation.}

\BH{get the relational invariant for each set of expressions as
  interpolant. Pre-formula: disjunction of cases, one for each
  expression. Post-formula: conjunction of rule constraints for all
  steps left. Note that we don't need to use the other relational
  invariants.}

%%% Local Variables: 
%%% mode: latex
%%% TeX-master: "p"
%%% End: 
