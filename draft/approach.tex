% give the real meat of the thing
\section{Technical Approach}
%
In this section, we describe our technical approach in detail.
%
In \autoref{sec:rel-invs}, we define the class of proof
structures synthesized by \sys.
%
In \autoref{sec:verifier}, we present the verification algorithm used
by \sys to synthesize such structures.

\subsection{Relational invariants as equivalence proofs}
\label{sec:rel-invs}
%
\sys, given programs $P_0$ and $P_1$, attempts to synthesize a proof
that $P_0$ is equivalent to $P_1$ that is represented as set of
subexpressions of $P_0$ and $P_1$, each mapped to relational
invariants that relate all evaluations of the subexpressions.
%
For the rest of this section, let $P_0, P_1 \in \stlc$ be two fixed,
arbitrary $\stlc$ programs.

% define symbolic relations:
For each $i \in \elts{0, 1}$ and $n \in nats$, let $\symvarsidx{i}{n}$
denote a distinct set of symbolic variables.
%
\begin{defn}
  \label{defn:sym-rels}
  Let $R: \subexps(P_0)^{*} \times \subexps(P_1)^{*} \partto
  \formulas{ \bigunion_{i \in \elts{0 , 1}, n \in \nats }
    \symvarsidx{i}{n} }$ be such that %
  for all $E_0 \in \subexps(P_0)^{*}$ and $E_1 \in \subexps(P_1)^{*}$,
  $R(E_0, E_1) \in \formulas{ %
    \bigunion_{ 0 \leq i < |P_0|} \symvarsidx{0}{i} \union %
    \bigunion_{0 \leq j < |P_1|} \symvarsidx{1}{j} }$.
  %
  Then $R$ are \emph{symbolic relations}.
\end{defn}
%
The space of symbolic relations of $P_0$ and $P_1$ is denoted
$\symrels{P_0}{P_1}$.

% relational invariants
% define relational invariants
Relational invariants of $P_0$ and $P_1$ are a map from sets of
subexpressions of $P_0$ and $P_1$, with each set $S$ mapped to a
formula that relates the evaluations of all expressions in $S$.
%
Relational invariants soundly describe all steps of evaluation that
can be taken by $P_0$ and $P_1$.
\begin{defn}
  \label{defn:rel-invs}
  Let $R_0, R_1 \in \symrels{P_0}{P_1}$ be such that %
  % condition: relation of empty set is entailed by true:
  \textbf{(1)} $\true \entails R_0(\epsilon, \epsilon) \lor
  R_1(\epsilon, \epsilon)$;
  % condition: steps are valid:
  \textbf{(2)} for all $e_0 \in \subexps{P_0}^{*}$, %
  $E_0 \in \subexps{P_0}^{*}$, %
  $E_1 \in \subexps{P_1}^{*}$, and %
  each $r \in \evalrules$ that decomposes $e_0$ to $E_0' \in
  \subexps(P_0)^{*}$, %
  there are some $E_0'', E_0''' \in \subexps{P_0}^{*}$ such that $E_0'
  \concat E_0 = E_0'' \concat E_0'''$ and
  \begin{align*}
    R_0(E_0'' \concat E_0''', E_1), %
    \subs{ \symstep{P_0}{ r } }{ \vars[ E_0' ], \vars[ e_0 ] } 
    & \entails R_0(e_0 \cons E_0, E_1) \\
    %
    R_1(E_0'' \concat E_0''', E_1), %
    \subs{ \symstep{P_0}{ r } }{ \vars[ E_0' ], \vars[ e_0 ] }
    & \entails R_0(e_0 \cons E_0, E_1)
  \end{align*}
  % symmetric condition for P1
  \textbf{(3)} for all $E_0 \in \subexps{P_0}^{*}$, %
  $e_1 \in \subexps{P_1}$, %
  $E_1 \in \subexps{P_1}^{*}$, and %
  each $r \in \evalrules$ that decomposes $e_1$ to $E_1' \in
  \subexps(P_1)^{*}$, %
  there are some $E_1'', E_1''' \in \subexps{P_1}^{*}$ such that
  $E_1' \concat E_1 = E_1'' \concat E_1'''$ and
  \begin{align*}
    R_0(E_0, E_1'' \concat E_0'''), %
    \subs{ \symstep{P_1}{ r } }{ \vars[ E_0' ], \vars[ e_0 ] }
    & \entails R_1(E_0, e_1 \cons E_1) \\
    R_1(E_0, E_1'' \concat E_1'''), %
    \subs{ \symstep{P_1}{ r } }{ \vars[ E_1' ], \vars[ e_1 ] }
    & \entails R_1(E_0, e_1 \cons E_1) 
  \end{align*}
  %
  Then $R_0$ and $R_1$ are \emph{relational invariants} of $P_0$ and
  $P_1$.
\end{defn}
%
The space of relational invariants of $P_0$ and $P_1$ is denoted
$\relinvs{P_0}{P_1}$.

% strengthen relational invariants to evidence of equivalence:
If relational invariants $R$ are such that the invariant for $\elts{
  P_0, P_1 }$, combined with the assumption that $P_0$ and $P_1$ are
given equal values, entails that $P_0$ and $P_1$ evaluate to equal
values, then $R$ is evidence of the partial equivalence of $P_0$ and
$P_1$.
%
\begin{defn}
  \label{defn:eq-pf}
  Let $R \in \relinvs{P_0}{P_1}$ be such that $R(P_0, P_1), \alpha^0_0
  = \alpha^1_0 \entails \nu^0 = \nu^1$.
  %
  Then $R$ is \emph{evidence} of $P_0 \equiv P_1$.
\end{defn}

% example of relational invariants:
\begin{ex}
  \label{ex:rel-invs}
  \BH{give example relational invariants}
\end{ex}

% evidence is a valid proof of partial equivalence:
If there is evidence of the equivalence of $P_0$ and $P_1$, then $P_0
\equiv P_1$.
%
\begin{lemma}
  \label{lemma:equiv-evidence}
  If there are some $R \in \relinvs{P_0}{P_1}$ such that $R$ is
  evidence of $P_0 \equiv P_1$, then $P_0 \equiv P_1$.
\end{lemma}

% define class of invariants that are actually maintained
\sys, given programs $P_0$ and $P_1$ attempts to determine if $P_0
\equiv P_1$ by synthesizing evidence of $P_0 \equiv P_1$ from a map
from sets of derivations to relational invariants.
%
Let the space of such maps be denoted $\dersrels{P_0}{P_1} =
\subexps(P_0)^{*} \times \subexps(P_1)^{*} \to \symrels{P_0}{P_1}$.
%
\begin{defn}
  \label{defn:der-rel-invs}
  Let $I \in \dersrels{P_0}{P_1}$ be such that %
  \begin{enumerate}
  \item 
    % supports relational invariants of complete derivations:
    For all $D_0 \in \dertrees{P_0}$ and $D_1 \in \dertrees{P_1}$,
    \[ I([ D_0 ], [ D_1 ]), \alpha_0 = \alpha_1 \entails %
    \nu_0 = \nu_1 \]
  \item 
    % soundly models each step of P0:
    For all $d_0 \in \dertrees{P_0}$, %
    $D_0 \in \dertrees{P_0}^{*}$, %
    $D_1 \in \dertrees{P_1}^{*}$ such that $(d_0 \cons D_0, D_1) \in
    \domain(I)$, %
    there are some $D_0', D_0'' \in \dertrees{P_0}^{*}$ such that
    $\immsubtrees{d_0} \concat D_0 = D_0' \concat D_0''$ and %
    \[ I(D_0', D_1), I(D_0'', D_1), %
    \symstep{ \head{ \immsubtrees{ d_0 } } }{ d_0 } \entails %
    I(D_0, D_1) \]
  \item 
    % soundly models each step of P1:
    For all $D_0 \in \dertrees{P_0}^{*}$, %
    $d_1 \in \dertrees{P_1}$, %
    $D_1 \in \dertrees{P_1}^{*}$ such that $(D_0, d_1 \cons D_1) \in
    \domain(I)$, %
    there are some $D_1', D_1'' \in \dertrees{P_1}^{*}$ such that
    $\immsubtrees{d_1} \concat D_1 = D_1' \concat D_1''$ and %
    \[ I(D_0, D_1'), I(D_0, D_1''), %
    \symstep{ \head { \immsubtrees{ d_1 } } }{ d_1 } \entails %
    I(D_0, D_1) \]
  \end{enumerate}
  %
  Then $I$ are \emph{derivation relational invariants} of $P_0$ and
  $P_1$.
\end{defn}
%
The space of all derivation relational invariants for $P_0$ and $P_1$
is denoted $\dersinvs{P_0}{P_1}$.
%
\BH{add a running example}

% define inductive derivation relational invariants
Derivation relational invariants are inductive if they define
relational invariants.
%
In particular, for $I \in \dersinvs{P_0}{P_1}$, let $I' \in
\relinvs{P_0}{P_1}$ be such that for all $E_0 \in \subexps(P_0)^{*}$
and $e_1 \in \subexps(P_1)^{*}$,
\begin{align*}
  \exprel{I}(E_0, E_1) = %
  \biglor \setformer{ I(D_0, D_1)}{ %
    & d_0^0, \ldots, d_0^m \in \dertrees{P_0}^{*}, %
    \explabel{d_0^0}(\head{d_0^0}), \ldots,
    \explabel{d_0^m}(\head{d_0^m}) = E_0, \\
    & d_1^0, \ldots, d_1^n \in \dertrees{P_1}^{*},  %
    \explabel{d_1^0}(\head{d_1^0}), \ldots,
    \explabel{d_1^n}(\head{d_1^n}) = E_1 }
\end{align*}
%
Derivation invariants $I$ are inductive for $P_0$ and $P_1$ if they
are the union of two derivation relations that define relational
invariants of $P_0$ and $P_1$.
%
\begin{defn}
  \label{defn:inductive-der-rels}
  For $I_0, I_1 \in \dersrels{P_0}{P_1}$, if $(\exprel{I_0},
  \exprel{I_1})$ are relational invariants of $P_0$ and $P_1$, then
  $I_0 \union I_1$ are \emph{inductive} derivation relational
  invariants.
\end{defn}
%
Derivation relational invariants are evidence of partial equivalence,
by \autoref{lemma:equiv-evidence}.
%
\sys, given programs $P_0$ and $P_1$ attempts to determine their
partial equivalence by synthesizing inductive derivation relational
invariants.

\BH{integrate this, define space of cuts, define space of tilings}
%
Let $\mathcal{C}_0 \subseteq \nodes{D_0}^{*}$ and $\mathcal{C}_1
\subseteq \nodes{D_1}^{*}$ be the cuts of $D_0$ and $D_1$,
respectively, and let $\mathcal{C}_0' \subseteq \nodes{D_0}$ and
$\mathcal{C}_1' \subseteq \nodes{D_1}$ be the subsequences of
$\mathcal{C}_0$ and $\mathcal{C}_1$, respectively.
%
For $C \in \mathcal{C}_0$ and $T \subseteq \mathcal{C}_0$, if $C$ is
the union of $T$, then $T$ is a \emph{tiling} of $C$, and similarly
for $\mathcal{C}_1$.

\subsection{Verification algorithm}
\label{sec:verifier}
\begin{figure}
  \centering
\begin{algorithm}[H]
  % Declare IO markers.
  \SetKwInOut{Input}{Input}
  %
  \SetKwInOut{Output}{Output}
  % Declare sub-program (procedure) markers.
  \SetKwProg{myproc}{Procedure}{}{}
  % Inputs: a program
  \Input{Programs $\cc{P}_0, \cc{P}_1 \in \stlc$.}
  % Output: decision for safety
  \Output{A decision as to whether $\cc{P}_0 \equiv \cc{P}_1$.}
  % verify: main procedure
  \myproc{$\verify(\cc{P}_0, \cc{P}_1)$ \label{line:core-begin}} %
  { \myproc{$\verifyaux(I)$ \label{line:core-aux-begin} }{ %
      % check if the solution is inductive:
      \Switch{$\chkinductive(I)$ \label{line:core-chkind} }{ %
        \lCase{$\isind$}{ %
          % case: it is. Return that P0, P1 are equivalent.
          \Return{$\true$} \label{line:core-ret-equiv} %
        } %
        \Case{$D_0 \in \dertrees{ \cc{P}_0 }$, %
          $D_1 \in \dertrees{ \cc{P}_1 }$ %
          \label{line:core-case-cex}}{ %
          \Switch{$\verifyders(D_0, D_1)$ \label{line:core-vders} }{ %
            \lCase{$\nonequiv$: \label{line:core-subcase-non-equiv} }{ %
              \Return{$\false$} \label{line:core-ret-non-equiv} } %
            \lCase{$I' \in \dersrels{P_0}{P_1}$: %
              \label{line:core-subcase-invs} }{ %
              \Return{ $\verifyaux(\mergeinvs(I, I'))$ } %
              \label{line:core-recurse}
            } %
          } %
        } %
      } %
    } \label{line:core-aux-end} %
    \Return{$\verifyaux(\emptyset)$} \label{line:core-base} } %
  %
  \caption{% interface of verify:
    \verify: an equivalence verifier.
  }
  \label{alg:verify}
\end{algorithm}
\end{figure}

% verifier top-level:
The equivalence verifier \verify is given in \autoref{alg:verify}.
%
\verify, given $P_0, P_1 \in \stlc$ (\autoref{line:core-begin}),
defines a procedure $\verifyaux$ that, given derivation relational
invariants $I$, attempts to determine if $P_0 \equiv P_1$ by finding
inductive derivation relational invariants built from $I$
(\autoref{line:core-aux-begin}---\autoref{line:core-aux-end}).
%
\verify calls \verifyaux on the empty map of derivation relational
invariants and returns the result (\autoref{line:core-base}).

% verify auxiliary procedure:
\verifyaux, given derivation relational invariants $I$, runs procedure
\chkinductive on $I$ (\autoref{line:core-chkind}; 
%
\chkinductive is given in \autoref{sec:chk-ind}).
%
If \chkinductive determines that some restriction of $I$ are
inductive, then \chkinductive returns the value $\isind$, and
\verifyaux returns that $P_0 \equiv P_1$
(\autoref{line:core-ret-equiv}).

% case: no inductive restriction found:
Otherwise, \chkinductive returns a derivation $D_0$ of $P_0$ and $D_1$
of $P_1$ that do not have derivation relational
invariants in $I$ (\autoref{line:core-case-cex}).
%
\verifyaux then runs a procedure \verifyders on $D_0$ and $D_1$
(\autoref{line:core-vders}; \verifyders is given in
\autoref{sec:verify-ders}).
%
If \verifyders returns the value $\nonequiv$ to denote that $D_0$ and
$D_1$ are not equivalent (\autoref{line:core-subcase-non-equiv}), then
\verifyaux returns that $P_0 \not\equiv P_1$
(\autoref{line:core-ret-non-equiv}).

% subcase: finds path-pair invariants:
Otherwise, \verifyders returns relational derivation invariants $I'$
of $D_0$ and $D_1$ (\autoref{line:core-subcase-invs}).
%
\verifyaux runs a procedure \mergeinvs on $I$ and $I'$, which
generates derivation relational invariants of all pairs of derivations
with relational invariants in both $I$ and $I'$.
%
\verifyaux recurses on the generated invariants and returns the result
(\autoref{line:core-recurse}).

% mergeinvs: combine using disjunction:
\mergeinvs, given derivation relational invariants $I, I' \in
\dersinvs{P_0}{P_1}$, generates $I'' \in \dersinvs{P_0}{P_1}$ such
that for each $D_0 \in \dertrees{P_0}$ and $D_1 \in \dertrees{P_1}$, %
\textbf{(1)} if $(D_0, D_1) \in \domain(I) \setminus \domain(I')$,
then $I''(D_0, D_1) = I(D_0, D_1)$;
%
\textbf{(2)} if $(D_0, D_1) \in \domain(I') \setminus \domain(I)$,
then $I''(D_0, D_1) = I'(D_0, D_1)$; %
\textbf{(3)} otherwise, $I''(D_0, D_1) = I(D_0, D_1) \lor I'(D_0,
D_1)$.

\subsubsection{Determining inductiveness of derivation invariants}
\label{sec:chk-ind}
% algorithm for checking inductiveness:
\begin{figure}
  \centering
  \begin{algorithm}[H]
    % Declare IO markers.
    \SetKwInOut{Input}{Input}
    % 
    \SetKwInOut{Output}{Output}
    % Declare sub-program (procedure) markers.
    \SetKwProg{myproc}{Procedure}{}{}
    % Inputs: a program
    \Input{$D \in \dersinvs{P_0}{P_1}$.}
    % Output: decision for safety
    \Output{$\isind$ to denote that some restriction of $D$ is
      inductive, or derivations of $P_0$ and $P_1$ not defined in $D$.}
    % verify: main procedure
    \myproc{$\chkinductive(D)$ \label{line:chkind-begin}} %
    { \myproc{$\chkindaux(n, I)$ \label{line:chkind-aux-begin} } %
      { % check given node:
        \lIf{$\dersexps{D}(n) = (\epsilon, \epsilon)$ %
          \label{line:chkind-is-empty} }{
          % case: node is mapped to empty sets of expressions:
          \Return{$\true$}
        }
        \lElseIf{$I(\dersexps{D}(n)) \entails \dersrelinvs{D}(n)$ %
          \label{line:chkind-is-entailed} }
        {
          % case: node is discharged:
          \Return{$\true$}
        }
        \lElse{
          \Return{$\chooseres(\unwind(n, D, I, P_0), %
            \unwind(n, D, I, P_1))$ \label{line:chkind-unwind} } %
        }
      \label{line:chkind-aux-end} } %
    \Return{$\chkindaux(\head{D}, \lambda x.\ \true)$} %
    \label{line:chkind-base} } %
    % 
    \caption{% interface of chkinductive:
      \chkinductive: determines the inductiveness of derivation
      relational invariants. }
    \label{alg:verify-ders}
  \end{algorithm}
\end{figure}
%
The algorithm \chkinductive for checking the inductiveness of
derivation relational invariants is given in
\autoref{alg:verify-ders}.
%
\chkinductive, given derivation relational invariants $D$
(\autoref{line:chkind-begin}), defines a procedure \chkindaux that
takes $n \in \nodes(D)$ and $I: \subcuts{P_0}{P_1} \to
\symrels{P_0}{P_1}$ and returns either %
\textbf{(1)} $\isind$ to denote that if $I$ are inductive relational
invariants, then $\dersexps{D}(n)$ have inductive relational
invariants that entail $I$ or
%
\textbf{(2)} derivations of the expressions in $\dersexps{D}(n)$ that
are not defined in $D$
(\autoref{line:chkind-aux-begin}---\autoref{line:chkind-aux-end}).
%
If case \textbf{(1)} holds, we say that $n$ in $D$ is inductive under
$I$.
%
\chkinductive calls \chkindaux on the head of the $D$ and a map from
each subcut pair to $\true$, and returns the result
(\autoref{line:chkind-base}).

\BH{walk through algorithm}
% auxiliary function:
\chkindaux, given $n \in \nodes{D}$, $I : \subcuts{P_0}{P_1}$,
tests if $n$ is mapped to the pair of empty subcuts, and if so,
returns $\isind$ (\autoref{line:chkind-is-empty}).
%
Otherwise, \chkindaux tests if the relational invariant of the
expressions of $n$ under $I$ entails the relational invariant of $n$,
and if so, returns $\isind$ (\autoref{line:chkind-is-entailed}).
%
Otherwise, \chkindaux runs the procedure \unwind on $n$, $D$, $I$,
and $P_0$, which returns either \textbf{(1)} $\isind$ to denote that
under each evaluation rule that can be applied to the subcut of $P_0$
%
\BH{define applicability of rule to subcut}
%
in $\dersexps{D}(n)$, there is some tiling that is inductive under
$I$ or %
\textbf{(2)} returns derivations of $\dersexps{D}(n)$ that are not
defined in $D$;
%
the implementation of \unwind is defined in detail below.
%
\chkindaux runs \unwind similarly on $n$, $D$, $I$, and $P_1$.

% chooseres:
\chkindaux runs \chooseres on the results of the two calls.
%
If either call to \unwind returns $\isind$, then \chooseres returns
$\isind$.
%
Otherwise, \chooseres returns some set of derivations returned by one
of the calls to \unwind.

\paragraph{Implementation of \unwind}
\BH{complete}

\subsubsection{Verifying equivalence of derivations}
\label{sec:verify-ders}
\begin{figure}
  \centering
  \begin{algorithm}[H]
    % Declare IO markers.
    \SetKwInOut{Input}{Input}
    % 
    \SetKwInOut{Output}{Output}
    % Declare sub-program (procedure) markers.
    \SetKwProg{myproc}{Procedure}{}{}
    % Inputs: a program
    \Input{$D_0 \in \dertrees{P_0}$, $D_1 \in \dertrees{P_1}$, %
      $T \in \tilings{D_0}{D_1}$.}
    % Output: decision for safety
    \Output{Derivation relational invariants of $D_0$ and $D_1$ or the
      value $\nonequiv$.}
    % verify: main procedure
    \myproc{$\verifyders(D_0, D_1, T)$ \label{line:vders-begin}} %
    { \myproc{$\verifydersaux(T')$ \label{line:vders-aux-begin} } %
      { % construct CHC for pair of derivation and tilings:
        $\mathcal{S} \assign \derschc(D_0, D_1, T')$ %
        \label{line:vders-cons-chc} \;
        % try to solve system:
        \Switch{$\solvechc(\mathcal{S})$ %
          \label{line:vders-solve-chc} }{ %
          \lCase{$\sigma \in \interps{ \mathcal{S} }$ %
            \label{line:vders-case-soln} }{ %
            % case: system is solvable. Return solution:
            \Return{$(T, \sigma)$} \label{line:vders-ret-invs} %
          } %
          \Case{$D \in \ders{ \mathcal{S} }$ %
            \label{line:vders-case-cex}}{ %
            % check if D is valid counter-example:
            \Switch{$\isvalidcex(D)$ \label{line:vders-chk-cex}}{
              \lCase{$\isvalid$ \label{line:vders-subcase-valid} }{
                % subcase: the counterexample is valid:
                \Return{$\nonequiv$} \label{line:vders-ret-nonequiv} } %
              \lCase{$C \in \mathcal{S}$ %
                \label{line:vders-subcase-extra} }{ %
                \Return{ $\verifydersaux(\remove{ T' }{ \decomp(C) })$ %
                  \label{line:vders-recurse} } %
              }
            } %
          } %
        } %
        \label{line:vders-aux-end} } %
      \Return{ $\verifydersaux(T)$ } \label{line:vders-base} \;
    } %
    % 
    \caption{% interface of verifyders:
      \verifyders: given a pair of derivations, determines their
      equivalence, using procedures \solvechc and \isvalidcex, given
      in \autoref{sec:verify-ders}.
    }
    \label{alg:verify-ders}
  \end{algorithm}
\end{figure}

%
\verifyders (\autoref{alg:verify-ders}), given $D_0 \in
\dertrees{P_0}$, $D_1 \in \dertrees{P_1}$, and $T \in
\tilings{D_0}{D_1}$, returns either \textbf{(1)} relational derivation
invariants of $D_0$ and $D_1$ with a tiling relation contained by $T$
or %
\textbf{(2)} the value $\nonequiv$ to denote that $D_0 \not\equiv
D_1$.
%
\verifyders reduces the problem of determining $D_0 \equiv D_1$ to
solving a series of recursion-free systems of Constrained Horn Clauses
(\autoref{sec:chcs}).

% introduce aux function:
$\verifyders$ defines a procedure $\verifydersaux$
(\autoref{line:vders-aux-begin}---\autoref{line:vders-aux-end}) that,
given tilings $T' \in \tilings{D_0}{D_1}$, either returns derivation
relational invariants of $D_0$ and $D_1$ with a tiling contained by
$T'$ or the value $\nonequiv$ to denote that $D_0$ and $D_1$ have no
such relational invariants.
%
$\verifyders$ calls $\verifydersaux$ on $T$ and returns the result
(\autoref{line:vders-base}).

% walk through verifydersaux
\verifydersaux, given tilings $T'$ (\autoref{line:vders-cons-chc}),
constructs a CHC system $\mathcal{S}$ for which each solution defines
relational invariants that combined, with $T'$, are derivation
relational invariants of $D_0$ and $D_1$ by running the procedure
$\derschc$ on $D_0$, $D_1$, and $T'$ (\autoref{line:vders-cons-chc};
%
the definition of $\derschc$ is given below).
%
\verifydersaux then runs \solvechc on $\mathcal{S}$
(\autoref{line:vders-solve-chc}).
% case: system has solution:
If \solvechc returns a solution $\sigma \in \interps{ \mathcal{S} }$
(\autoref{line:vders-case-soln}), then \verifydersaux returns $(T,
\sigma)$ as derivation relational invariants of $D_0$ and $D_1$.

% case: system has counter-derivation:
Otherwise, if \solvechc returns a counter-derivation $D \in \ders{
  \mathcal{S} }$ (\autoref{line:vders-case-cex}), then \verifydersaux
determines if $D$ is valid evidence that $D_0 \not\equiv D_1$ by
running a procedure \isvalidcex on $D$ (\autoref{line:vders-chk-cex};
%
\isvalidcex is defined below).
% subcase: counterexample is valid:
If \isvalidcex returns the value $\isvalid$, then \verifydersaux
returns $\nonequiv$.

% subcase: counterexample is not valid:
Otherwise, \isvalidcex returns $C \in \mathcal{S}$ such that each
solution to $\remove{T'}{C}$ defines derivation relational invariants
of $D_0$ and $D_1$ (\autoref{line:vders-subcase-extra}).
%
\verifydersaux recurses on $\remove{T'}{C}$ and returns the result
(\autoref{line:vders-recurse}).

% give DersCHC:
\paragraph{Relational Invariants as CHC solutions}
% introduce tiling relation:
The CHC system $\mathcal{S} = \derschc(D_0, D_1, T')$ is defined as
follows.
% define relational predicates per tiling:
The relational predicates of $\mathcal{S}$ are $\cuts{D_0} \times
\cuts{D_1}$.

% define clauses per tiling:
The clauses of $\mathcal{S}$ are defined as follows.
%
$\mathcal{S}$ contains clauses $\stepclauses_0$ that model all steps
of evaluation taken by $P_0$.
%
In particular, for each $n' \in \nodes{D_0}$ with children $N \in
\nodes{D_0}^{*}$, %
$C_0 \in \cuts{D_0}$ such that $n' \cons C_0 \in \cuts{D_0}$,
$C_1 \in \cuts{D_1}$, there is a clause %
\[ (N \concat C_0, C_1)[ \symvars{N}, \symvars{C_0}, \symvars{C_1} ], %
\derstep{D_0}{ n' } \entails %
(n' \cons C_0, C_1) \]
%
\BH{define shorthand, symbolic variables used}
%
$\mathcal{S}$ contains clauses $\stepclauses_1$ that model all steps
of evaluation taken by $P_1$.
%
In particular, for each $n' \in \nodes{D_1}$ with children $N \in
\nodes{D_1}^{*}$, %
$C_0 \in \cuts{D_0}$, and %
$n' \in \nodes{C_1}$, %
$C_1 \in \cuts{D_1}$ such that $n' \cons C_1 \in \cuts{D_1}$, %
there is a clause %
\[ (C_0, N \concat C_1)[ \symvars{C_0}, \symvars{N}, \symvars{C_1} ], %
\derstep{D_1}{ n' } \entails %
(n' \cons C_0, C_1) \]
%
For each tiling in $T'$, there is a clause in the set $\decompclauses$
that constrains that the combination of invariants of a tiling entail
the invariant of the pair of cuts that they tile.
%
I.e., for each $\mathcal{M}_0 \subseteq \cuts{D_0}$, %
$\mathcal{M}_1 \subseteq \cuts{D_1}$, %
$C_0 \in \cuts{D_0}$, and %
$C_1 \in \cuts{D_1}$, there is a clause
%
\[ \setformer{ M_0[ \symvars{ M_0 } ]}{ M_0 \in \mathcal{M}_0 }, %
\setformer{ M_1[ \symvars{ M_1 } ] }{ M_1 \in \mathcal{M}_1 } \entails %
(C_0[ \symvars{C_0} ], C_1[ \symvars{C_1} ])
\]
%
The query node of $\mathcal{S}$ is $( [ \head{D_0} ], [ \head{D_1}
])$.
%

\paragraph{Checking validity of a counter-derivation as a
  counterexample to equivalence}
%
The procedure \isvalidcex, given $D_0 \in \dertrees{P_0}$, $D_1 \in
\dertrees{P_1}$, $T \in \tilings{D_0}{D_1}$ and $D \in \ders{
  \mathcal{S} = \derschc(D_0, D_1, T)}$, collects the union $D'
\subseteq \mathcal{S}$ of the instances
%
\BH{background: define instances of clauses in a CHC derivation}
%
of clauses $\decompclauses \subseteq \mathcal{S}$ that occur in $D$.
%
If $D' = \emptyset$, then \isvalidcex returns $\isvalid$ to denote
that $D_0 \not\equiv D_1$.
%
Otherwise, \isvalidcex returns an element in $D'$.

%%% Local Variables: 
%%% mode: latex
%%% TeX-master: "p"
%%% End: 
