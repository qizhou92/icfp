\section{Background}
\label{sec:background}
%
In this section, we define as our target language an extension of the
simply-typed $\lambda$-calculus (STLC) (\autoref{sec:stlc}).
%
We then define partial equivalence of STLC programs
(\autoref{sec:peq}).

\subsection{A subject functional language}
\label{sec:stlc}
\begin{figure}[t]
  % structural types,
  \begin{minipage}{0.36\linewidth}
    \begin{align*} %
      \sigma \rightarrow &\ \unitty\ |\ \basety\ |\ X \\
      | &\ \sigma \rightarrow \sigma \\
      | &\ \sigma \times \sigma \\
      | &\ \sigma + \sigma \\
      | &\ \mu X. \sigma
    \end{align*}
    \caption{The space of structural types, $\types$.
      % 
      Metavariable $\sigma$ ranges over $\types$ and $X$ ranges over
      least-fixed point variables.} %
    \label{fig:types}
  \end{minipage}
  \qquad
  % expressions,
  \begin{minipage}{0.55\linewidth}
    \begin{align*}
      % expression is the unit,
      e \rightarrow &\ \unit\ 
                      % a constant,
                      |\ c\
                      % or a variable,
                      |\ a\ 
                      % or a base operation,
                      |\ e_0 \oplus e_1 \\
      % or an abstraction,
      | &\ \abs{ x }{ \sigma }{ e } \
      % or an application,
          | \ e_0\ e_1 \\
      % or a pairing,
      | &\ \pair{e_0}{e_1} \
      % or a let-expression,
          |\ \letexp{x}{\sigma_x }{y}{\sigma_y }{e_0}{e_1} \\
      % or an injection or a match,
      | &\ \injl\ e\ |\ \injr\ e\ \\
      | &\ \matchexp{e_0}{x}{\sigma_x}{e_1}{y}{\sigma_y}{e_2} \\
      % or a fix
      | &\ \fix{f}{\sigma}{e}
    \end{align*}
    \caption{The space of program expressions, $\expr$.
      %
      Metavariable $e$ ranges over $\expr$, %
      $c$ ranges over $\consts$, %
      $\oplus$ ranges over base operations, %
      $x$, $y$ and $f$, range over $\vars$, and %
      $\sigma$ ranges over structural types. } %
    \label{fig:exprs}
  \end{minipage}
  \qquad
\end{figure}

% define base types:
\paragraph{Structural types}
%
A structural type defines the structure of values.
%
The space of types is defined in \autoref{fig:types}.
%
A type is the unit, the base type, a type variable, a function type, a
sum of types, a product of types, or a least-fixed point of a type.
%
A type $\sigma$ is well-formed if each type variable $X$ occurs only
as a component of a type of the form $\mu X. \sigma'$ that itself is a
component of $\sigma$.

% define expression syntax:
\paragraph{Expressions}
% define spaces,
\autoref{fig:exprs} defines a space of expressions over spaces of
constants $\consts$ and variables $\vars$.
% define construction of expressions,
An expression is the unit, a constant, a variable, a base operation
over expressions, an abstraction, an application, a pair, a
\cc{let}-binding, a left injection, a right injection, a match, or a
fixpoint.
%
For each $e \in \expr$, the \emph{free-variables} of $e$ are denoted
$\freevars(e)$.

% define semantics:
\paragraph{Semantics}
%
A value is the unit, a constant, an abstraction, a pair of values, a
left-injection of a value, or a right-injection of a value;
%
the space of values is denoted $\values$.
%
The space of base values
$\basevals = \consts \union \elts{ \injl\ \cc{()}, \injr\ \cc{()}}$
contains the space of constants and the left injections of the unit,
which will be used to model Boolean values.
%
For each base operation $\oplus$, there is a semantic function
$\interp{\oplus} : \basevals \times \basevals \to \basevals$.

\begin{figure}
  \centering
  \begin{gather*}
    % unit,
  \inference[ UNIT ]{ }{ \unit \evalsto \unit } 
  % evaluate const:
  \inference[ CONST ]{ }{ c \evalsto c } 
  % evaluate binary operations:
  \inference[ OP ]{ %
    e_0 \evalsto c_0 & %
    e_1 \evalsto c_1 }{ %
    e_0 \oplus e_1 \evalsto c_0 \interp{\oplus} c_1 } \\
  % abstraction, 
  \inference[ ABS ]{ }{ \abs{ x }{ \sigma }{ e } \evalsto %
    \abs{ x }{ \sigma }{ e } \abs{x}{\sigma}{e} } 
  % application,
  \inference[ APP ]{ %
    e_0 \evalsto \abs{x}{\sigma}{e_0'} & %
    e_1 \evalsto v_1 &
    \replace{ e_0' }{ x }{ v_1 } \evalsto v }{ %
    e_0\ e_1 \evalsto v } \\
  % pair,
  \inference[ PAIR ]{ %
    % evaluate components,
    e_0 \evalsto v_0 & e_1 \evalsto v_1 }{ %
    % pair them up to get final result
    \pair{ e_0 }{ e_1 } \evalsto (v_0, v_1) } 
  % let,
  \inference[ LET ]{
    % evaluate bound expression,
    e_0 \evalsto (v_0^l, v_0^r) &
    % evaluate result with values replaced
    \replace{ e_1 }{ x, y }{ v_0^l, v_1^r } \evalsto v_1 }{ %
    \letexp{x}{\sigma_x}{y}{\sigma_y}{e_0}{e_1} \evalsto v_1 } \\
  % inject left,
  \inference[ INJL ]{ e \evalsto v }{ %
    \injl\ e \evalsto \injl\ v } 
  % match left injection,
  \inference[ MATCHL ]{
    % evaluate target exp to a left injection,
    e_0 \evalsto \injl\ v_0 & %
    \replace{ e_1 }{ x }{ v_0 } \evalsto v_1 }{ %
    \matchexp{ e_0 }{ x }{\sigma_x}{ e_1 }{ y }{\sigma_y}{ e_2 } \evalsto v_1 } \\
  % inject right,
  \inference[ INJR ]{ e \evalsto v }{ \injr\ e \evalsto \injr\ v }
  % match right injection,
  \inference[ MATCHR ]{
    % evaluate target exp to a left injection,
    e_0 \evalsto \injr\ v_0 &
    \replace{ e_2 }{ y }{ v_0 } \evalsto v_2 }{ %
    \matchexp{ e_0 }{ x }{ e_1 }{ y }{ e_2 } \evalsto v_2 } 
  % fix:
  \inference[ FIX ]{ \replace{e}{ f }{ \fix{ f }{\sigma}{ e } } \evalsto v } %
  { \fix{ f }{\sigma}{ e } \evalsto v }
  \end{gather*}
  \caption{The natural semantics of expressions.
    %
    Metavariable $e$ ranges over $\expr$, %
    $c$ ranges over $\consts$, %
    $\oplus$ ranges over base operations, %
    $x$, $y$, and $f$ range over $\vars$, and
    $\sigma$ ranges over structual types.}
  \label{fig:nat-sem}
\end{figure}
%
For each $e \in \expr$ and $v \in \values$, the fact that $e$
\emph{evaluates} to $v$ is denoted $e \evalsto v$.
% 
The evaluation relation is a natural (i.e., \emph{big-step})
semantics, defined inductively by inference rules given in
\autoref{fig:nat-sem}.


% standard defns for logic:
\paragraph{Logic}
% define theory:
\sys infers types that relate evaluations of expressions using
formulas in formal logic.
%
Let the space of \emph{refinement values} be
$\fovals = \consts \union \elts{ \injl\ \unit, \injr\ \unit }$.
%
Let the \emph{refinement theory} $\mathcal{T}$ be a logical theory
with equality with a vocabularly whose constants are the program
constants, $\injl\ \unit$, and $\injr\ \unit$, and whose binary
functions are the space of binary operators.
%
Each model of $\mathcal{T}$ is isomorphic to the domain $\fovals$ and
the semantic functions $\interp{\oplus}$ for each binary operation
$\oplus$.
%
For practical languages modeled by $\expr$, $\mathcal{T}$ may be,
e.g., the quantifier-free fragment of the theory of linear arithmetic,
where $\consts$ is the space of integers, $\injl\ \unit$ and
$\injr\ \unit$ model true and false, and the space of binary operators
contains plus and the less-than-or-equal comparitor.

% define spaces of formulas:
For each space of logical variables $X$, the space of $\mathcal{T}$
formulas over $X$ is denoted $\formulas{ X }$.
%
For each formula $\varphi \in \formulas{X}$, the set of variables that
occur in $\varphi$ (i.e., the \emph{vocabulary} of $\varphi$) is
denoted $\vocab(\varphi)$.
% define models, satisfaction, entailment
For formulas $\varphi_0, \ldots, \varphi_n, \varphi \in \formulas{X}$,
the fact that $\varphi_0, \ldots, \varphi_n$ \emph{entail} $\varphi$
is denoted $\varphi_0, \ldots, \varphi_n \entails \varphi$.

% Define replacement and substitution:
For all vectors of variables $X = [ x_0, \ldots, x_n ]$ and
$Y = [ y_0, \ldots, y_n ]$, the $\mathcal{T}$ formula constraining the
equality of each element in $X$ with its corresponding element in $Y$,
i.e., the formula $\bigland_{0 \leq i \leq n} x_i = y_i$, is denoted
$X = Y$.
%
The repeated replacement of variables $\varphi[ y_0 / x_0 \ldots y_{n}
/ x_{n} ]$ is denoted $\replace{\varphi}{Y}{X}$.
%
For each formula $\varphi$ defined over free variables $X$,
$\replace{\varphi}{X}{Y}$ is denoted alternatively as $\varphi[Y]$.
%
The number of free variables in formula $\varphi$ is denoted
$\degreeof(\varphi)$.

\subsection{Constrained Horn Clauses}
\label{sec:chcs}

\subsubsection{Structure}
% definition of CHC
A Constrained Horn Clause is a body, consisting of a conjunctive set
of uninterpreted relational predicates and a constraint, and a head
relational predicate.
%
Relational predicates are predicate symbols paired with a map from
each symbol to its arity.
%
\begin{defn}
  \label{defn:rel-preds}
  For each space of symbols $\mathcal{R}$ and function $a: \mathcal{R}
  \to \nats$, $(\mathcal{R}, a)$ is a space of \emph{relational
    predicates}.
\end{defn}
%
The space of all relational predicates is denoted $\relpreds$.
%
For each space of relational predicates $\mathcal{R} \in \relpreds$,
we denote the predicate symbols and arity of $\mathcal{R}$ as
$\relsof{\mathcal{R}}$ and $\arityof{\mathcal{R}}$, respectively.
%
For relational predicates $\mathcal{R} \in \relpreds$ and symbol $R$,
we denote $R \in \relsof{ \mathcal{R} }$ alternatively as $R \in
\mathcal{R}$.
%
All definitions introduced in this section are given over a fixed,
arbitrary set of relational-predicate symbols $\mathcal{R}$.

% define applications of relational predicates:
An application of a relational predicate is a relational-predicate
symbol $R$ paired with a sequence of variables of length equal to the
arity of $R$.
%
\begin{defn}
  \label{defn:pred-apps}
  For $R \in \mathcal{R}$ and sequence of variables $Y \in X^{*}$ such
  that $|Y| = \arityof{\mathcal{R}}(R)$, $(R, Y)$ is an
  \emph{application}.
\end{defn}
%
The space of applications is denoted $\apps{\mathcal{R}}$.
%
For each application $A \in \apps{\mathcal{R}}$, the predicate symbol
and argument sequence of $A$ are denoted $\relof{A}$ and $\argsof{A}$
respectively.

% define CHC:
A Constrained Horn Clause is a body of applications, a constraint, and
a head application.
%
\begin{defn}
  \label{defn:chcs-structure}
  For $\mathcal{A} \subseteq \apps{\mathcal{R}}$ and %
  $\varphi \in \formulas{X}$, $\mathcal{B} = (\mathcal{A}, \varphi)$
  is a \emph{clause body}.
  % 
  For $H \in \apps{\mathcal{R}}$, $(\mathcal{B}, H)$ is a
  \emph{Constrained Horn Clause}.
\end{defn}
% define space of bodies:
The space of clause bodies is denoted $\bodies{ \mathcal{R} } =
\pset(\apps{\mathcal{R}}) \times \formulas{X}$.
%
For each $\mathcal{B} \in \bodies{ \mathcal{R} }$, the constraint of
$\mathcal{B}$ is denoted $\ctrof{ \mathcal{B} }$.
% define space, projection:
The space of Constrained Horn Clauses is denoted $\chc{ \mathcal{R} }
= \bodies{ \mathcal{R} } \times \apps{ \mathcal{R} }$.
%
For each clause $\mathcal{C} \in \chc{\mathcal{R}}$, the body and head
of $\mathcal{C}$ are denoted $\bodyof{ \mathcal{C} }$ and $\headof{
  \mathcal{C} }$, respectively.
% define siblings:
For each $\mathcal{S} \subseteq \chc{ \mathcal{R} }$, $\mathcal{C} \in
\mathcal{S}$, and all applications $A_0, A_1 \in \appsof{ \bodyof{
    \mathcal{C} } }$, $\relof{A_0}$ and $\relof{A_1}$ are
\emph{siblings} in $\mathcal{S}$.

% give conditions on normalization:
Let $\mathcal{S} \subseteq \chc{ \mathcal{R} }$ that is recursion-free
be such that there is exactly one relational predicate $\queryof{
  \mathcal{S} }$ that is the dependency of no relational predicate.
%
Then $\mathcal{S}$ is a \emph{normalized} recursion-free system.
%
For the remainder of this paper, we consider only normalized
recursion-free sets of CHCs, and denote the space of such sets as
$\chcs{ \mathcal{R} }$.

\subsubsection{Solutions}
\label{sec:chc-solns}
%
A solution to a CHC system $\mathcal{S}$ is an interpretation of each
relational predicate $R$ of arity $n$ as a formula over $n$ free
variables such that for each clause $\mathcal{C} \in \mathcal{S}$, the
conjunction of interpretations of all relational predicates in the
body of $\mathcal{C}$ and the constraint of $\mathcal{C}$ entail the
interpretation of the head of $\mathcal{C}$.
%
Let a map from each $R \in \relsof{ \mathcal{R} }$ to a formula over
an ordered vector of $\arityof{ \mathcal{R} }(R)$ free variables be an
\emph{interpretation} of $\mathcal{R}$;
%
let the space of interpretations of $\mathcal{R}$ be denoted
$\interps{ \mathcal{R} }$.
%
\begin{defn}
  \label{defn:chc-soln}
  For $\mathcal{B} \in \bodies{ \mathcal{R} }$ and $H \in \apps{
    \mathcal{R} }$, %
  let $\sigma \in \interps{ \mathcal{R} }$ be such that for each $R
  \in \mathcal{R}$, $\arityof{\mathcal{R}}(R) = \degreeof{\sigma(R)}$
  and %
  %
  \[ \elts{ \sigma(\relof{ A })[ \argsof{ A } ] }_{ A \in \appsof{
      \mathcal{B} } }, %
  \ctrof{ \mathcal{B} } \entails %
  \sigma( \relof{H} )[ \argsof{ H } ]
  \]
  Then $\sigma$ is a \emph{solution} of $(\mathcal{B}, H)$.
\end{defn}
% define partial solution 
For $\mathcal{S} \in \chcs{ \mathcal{R} }$, if %
\textbf{(1)} for each $\mathcal{C} \in \mathcal{S}$, $\sigma$ is a
solution of $\mathcal{C}$ and %
\textbf{(2)} $\sigma(\queryof{ \mathcal{S} }) \entails \false$, then
$\sigma$ is a \emph{solution} of $\mathcal{S}$.
%
The space of solutions of $\mathcal{S}$ is denoted $\solutions{
  \mathcal{S} }$.

% introduce solver:
A procedure that, given a recursion-free system $\mathcal{S}$ returns
either a solution of $\mathcal{S}$ or a value that denotes that
$\mathcal{S}$ has no solution is a CHC \emph{solver}.
%
\sys is defined as using a CHC solver, named $\solvechc$.

%%% Local Variables: 
%%% mode: latex
%%% TeX-master: "p"
%%% End: 
