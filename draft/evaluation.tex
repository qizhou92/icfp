\section{Evaluation}
\label{sec:evaluation}
%
In this section, we introduce several pairs of coding solutions,
translated to functional programs.
%
For each pair of programs $P_0$ and $P_1$, we identify an inductive
set of pairs of expressions $I$, and give a map from all pairs of
expressions in $I$ to relational invariants that prove the partial
equivalence of $P_0$ and $P_1$.

\paragraph{Solutions to Add Digits}
%
\NV{complete}

\paragraph{Solutions to Climbing Stairs}
%
\begin{figure}[t]
  \begin{minipage}{0.48\linewidth}
    \begin{Verbatim}[commandchars=\\\{\},codes={\catcode`\$=3\catcode`\^=7\catcode`\_=8}]
\PY{o}{(}\PY{k}{fun} \PY{n}{n} \PY{o}{\PYZhy{}\PYZgt{}}
   \PY{o}{(}\PY{k}{if} \PY{o}{(}\PY{n}{n} \PY{o}{\PYZlt{}}\PY{o}{=} \PY{l+m+mi}{1} \PY{c}{(*}\PY{c}{ 2 }\PY{c}{*)} \PY{o}{)} \PY{k}{then} \PY{o}{(}\PY{l+m+mi}{1} \PY{c}{(*}\PY{c}{ 3 }\PY{c}{*)}\PY{o}{)}
    \PY{k}{else}
       \PY{o}{(}\PY{o}{(}\PY{n}{fix} \PY{o}{(}\PY{k}{fun} \PY{n}{cS0Rec}
         \PY{o}{(}\PY{n}{sum}\PY{o}{,} \PY{n}{prev}\PY{o}{,} \PY{n}{cur}\PY{o}{,} \PY{n}{i}\PY{o}{)} \PY{o}{\PYZhy{}\PYZgt{}}
           \PY{o}{(}\PY{k}{if} \PY{n}{i} \PY{o}{\PYZlt{}} \PY{n}{n} \PY{c}{(*}\PY{c}{ 8 }\PY{c}{*)} \PY{k}{then}
               \PY{o}{(}\PY{n}{cS0Rec}
                 \PY{o}{(}\PY{n}{sum} \PY{o}{+} \PY{n}{prev}\PY{o}{,}
                  \PY{n}{cur}\PY{o}{,} \PY{n}{sum}\PY{o}{,} \PY{n}{i} \PY{o}{+} \PY{l+m+mi}{1} \PY{c}{(*}\PY{c}{ 11 }\PY{c}{*)}\PY{o}{)}
               \PY{c}{(*}\PY{c}{ 9 }\PY{c}{*)}\PY{o}{)}
            \PY{k}{else} \PY{o}{(}\PY{n}{sum} \PY{c}{(*}\PY{c}{ 10 }\PY{c}{*)}\PY{o}{)}
           \PY{c}{(*}\PY{c}{ 7 }\PY{c}{*)}\PY{o}{)}\PY{o}{)}
        \PY{c}{(*}\PY{c}{ 5 }\PY{c}{*)}\PY{o}{)}
           \PY{o}{(}\PY{l+m+mi}{2}\PY{o}{,} \PY{l+m+mi}{1}\PY{o}{,} \PY{l+m+mi}{0}\PY{o}{,} \PY{l+m+mi}{2} \PY{c}{(*}\PY{c}{ 6 }\PY{c}{*)} \PY{o}{)}
       \PY{c}{(*}\PY{c}{ 4 }\PY{c}{*)} \PY{o}{)}
   \PY{c}{(*}\PY{c}{ 1 }\PY{c}{*)} \PY{o}{)}
\PY{c}{(*}\PY{c}{ 0 }\PY{c}{*)} \PY{o}{)}
\end{Verbatim}

  \end{minipage}
  \begin{minipage}{0.48\linewidth}
    \input{code/ClimbStairs1.ml}    
  \end{minipage}
  %
  \caption{\cc{ClimbStairs0} and \cc{ClimbStairs1}: two solutions to
    the Climbing Stairs problem hosted on LeetCode, manually translated
    to OCaml.
    %
    All subexpressions are labeled with indices in comments.
    %
  }
  \label{fig:climb-stairs}
\end{figure}
%
\autoref{fig:climb-stairs} contains the pseudocode for two solutions
to the Climbing Stairs Problem hosted on the coding platform the
LeetCode.
% problem statement:
The Climbing Stairs Problem is to take an integer $n$ and return the
number of distinct sequences of steps that can be taken to climb $n$
steps, where in each step either one or two steps can be climbed. If
$n$ is less than or equal to one, then the solution is one.
%
\BH{walk through how the two solutions work}

%
\cc{climbStairs0} and \cc{climbStairs1} have the following proof of
partial equivalence:
%
\BH{put the pairs of subexpressions in a set}
%
\begin{align*}
  % pair of main subexpressions:
  (\cc{0}, \cc{0}) \mapsto & %
  \alpha_0^0 = \alpha_1^0 \implies \nu_0 = \nu_1 \\
  % pair of function bodies:
  (\cc{1}, \cc{1}) \mapsto & %
  \cc{n}_0 = \cc{n}_1 \implies \nu_0 = \nu_1 \\
  % pair of if guards:
  (\cc{2}, \cc{2}) \mapsto & %
  \cc{n}_0 = \cc{n}_1 \implies \nu_0 = \nu_1 \\
  % pair of then expressions:
  (\cc{3}, \cc{3}) \mapsto & \nu_0 = \nu_1 \\
  % pair of else expressions:
  (\cc{4}, \cc{4}) \mapsto & %
  \cc{n}_0 = \cc{n}_1 \implies \nu_0 = \nu_1 \\
  % pair of fixpoint functions:
  (\cc{5}, \cc{5}) \mapsto & %
  % inductive relational invariant (indices in step):
  (\cc{n}_0 = \cc{n}_1 \land %
  \alpha_0^0 = \alpha_1^1 \land %
  \alpha_0^1 = \alpha_1^1 \land %
  \alpha_0^3 + 1 = \alpha_1^2 \implies %
  \nu_0 = \nu_1) \land \\
  % for establishing final property:
  & (\cc{n}_0 = \cc{n}_1 \land %
  \alpha_0^0 = \alpha_0^1 + \alpha_1^1 \land %
  \alpha_0^1 = \alpha_1^1 \land %
  \alpha_0^3 = \alpha_1^2 \implies %
  \nu_0 = \nu_1) \\
  % pair of argument vectors:
  (\cc{6}, \cc{6}) \mapsto & %
  \pi_0(\nu_0) = \pi_0(\nu_1) + \pi_1(\nu_1) \land %
  \pi_1(\nu_0) = \pi_1(\nu_1) \land %
  \pi_3(\nu_0) = \pi_3(\nu_3) \\
  % fixpoint body:
  (\cc{7}, \cc{7}) \mapsto & 
  % inductive relational invariant (indices in step):
  (\cc{n}_0 = \cc{n}_1 \land %
  \cc{sum}_0 = \cc{count2}_1 \land %
  \cc{prev}_1 = \cc{count2}_1 \land %
  \cc{i}_3 + 1 = \cc{i}_1 \implies %
  \nu_0 = \nu_1) \land \\
  % for establishing final property:
  & (\cc{n}_0 = \cc{n}_1 \land %
  \cc{sum}_0 = \cc{count1}_1 + \cc{count2}_1 \land %
  \cc{prev}_1 = \cc{count2}_1 \land %
  \cc{i}_3 = \cc{i}_1 \implies %
  \nu_0 = \nu_1) \\
  % fixpoint body guards:
  (\cc{8}, \cc{8}) \mapsto & 0 \leq \cc{i}_1 - \cc{i}_0 \leq 1 \implies %
  \nu_0 = \nu_1 \\
  % fixpoint body then expressions:
  (\cc{9}, \cc{9}) \mapsto & 
  (\cc{n}_0 = \cc{n}_1 \land %
  \cc{sum}_0 = \cc{count2}_1 \land %
  \cc{prev}_0 = \cc{count2}_1 \land %
  \cc{i}_0 + 1 = \cc{i}_1 \implies %
  \nu_0 = \nu_1) \land \\
  % for establishing final property:
  & (\cc{n}_0 = \cc{n}_1 \land \cc{i}_0 \land %
  \cc{sum}_0 = \cc{count1}_1 + \cc{count2}_1 \land %
  \cc{prev}_1 = \cc{count2}_1 \land %
  \cc{i}_0 = \cc{i}_1 \implies %
  \nu_0 = \nu_1) \\
  % fixpoint body else expressions:
  (\cc{10}, \cc{10}) \mapsto & \cc{sum}_0 = \cc{count2}_1 \implies %
  \nu_0 = \nu_1 \\
  % argument steps preserve relative invariants over arguments:
  % subcase: CS1 is one step ahead of CS0:
  (\cc{11}, \cc{11}) \mapsto & 
  (\cc{sum}_0 = \cc{count2}_1 \land %
  \cc{prev}_1 = \cc{count2}_1 \land %
  \cc{i}_3 + 1 = \cc{i}_1 \implies \\
  & \pi_0(\nu_0) = \pi_1(\nu_1) \land %
  \pi_1(\nu_0) = \pi_0(\nu_1) \land %
  \pi_3(\nu_0) = \pi_2(\nu_1)) \land \\
  %
  & (\cc{sum}_0 = \cc{count1}_1 + \cc{count2}_1 \land %
  \cc{prev}_1 = \cc{count2}_1 \land %
  \cc{i}_0 = \cc{i}_1 \implies \\ %
  & \pi_0(\nu_0) = \pi_0(\nu_1) + \pi_1(\nu_1) \land %
  \pi_1(\nu_0) = \pi_1(\nu_1) \land %
  \pi_3(\nu_0) = \pi_2(\nu_1)) \\
\end{align*}
%
\BH{define notation for logical variables}
%
\BH{refactor entries to use common invariant meta-variables}

\BH{walk through relational invariants}

\paragraph{Solutions to Reverse}
%
\NV{complete}

\paragraph{Solutions to Trailing Zeroes}
\NV{complete}

%%% Local Variables: 
%%% mode: latex
%%% TeX-master: t
%%% End: 
