\section{Background}
\label{s:background}
%
In this section, we review the foundations on which \sys is built.
%
In \autoref{sec:logic}, we review definitions from formal logic; %
in \autoref{sec:rtlc}, we define a $\lambda$-calculus with refinement
types; %
in \autoref{sec:chcs}, we define systems of Constrained Horn Clauses.

% formal logic:
\subsection{Formal logic}
\label{sec:logic}
%
\sys uses logic to express and infer refinement types of programs.
%
All objects defined in this paper are defined over a fixed space of
logical variables $X$.
%
In particular, for theory $\mathcal{T}$, the space of $\mathcal{T}$
formulas over $X$ is denoted $\tformulas{\mathcal{T}}$.
%
For each formula $\varphi \in \tformulas{\mathcal{T}}$, the set of
variables that occur in $\varphi$ (i.e., the \emph{vocabulary} of
$\varphi$) is denoted $\vocab(\varphi)$.
% define models, satisfaction, entailment
For formulas $\varphi_0, \ldots, \varphi_n, \varphi \in
\tformulas{\mathcal{T}}$, the fact that $\varphi_0, \ldots, \varphi_n$
\emph{entail} $\varphi$ is denoted $\varphi_0, \ldots, \varphi_n
\entails \varphi$.

% Define replacement and substitution:
For sequences of variables $Y = [ y_0, \ldots, y_n ]$ and $Z = [ z_0,
\ldots, z_n ]$, the $\mathcal{T}$ formula constraining the equality of
each element in $Y$ with its corresponding element in $Z$, i.e., the
formula $\bigland_{0 \leq i \leq n} y_i = z_i$, is denoted $Y = Z$.
%
The repeated replacement of variables $\varphi[ z_0 / y_0 \ldots z_{n}
/ y_{n} ]$ is denoted $\replace{\varphi}{Y}{Z}$.
%
For each formula $\varphi$ defined over free variables $Y$,
$\replace{\varphi}{Y}{Z}$ is denoted alternatively as $\varphi[ Z ]$.
%
The number of free variables in formula $\varphi$ is denoted
$\degreeof{\varphi}$.

\subsection{A $\lambda$-calculus with refinement types}
\label{sec:rtlc}
\begin{figure*}[t]
  \begin{floatrow}[2]
    \ffigbox[.48\textwidth] %
    { \caption{The space of program expressions, $\expr$.} %
      \label{fig:exprs} }
    {  \begin{align*}
        % expression is a constant,
        e \rightarrow &\ c \\
        % or a variable,
        | &\ x \\
        % or a binary operation,
        | &\ e \oplus e \\
        % or a constructor application,
        | &\ C(e_0, \ldots, e_k) \\
        % or case split,
        | &\ \cc{match}\ e\ \cc{with}\ %
        C_0(\overline{x}_0)\ \cc{->}\ e_0 %
        \ldots %
        C_k(\overline{x}_k)\ \cc{->}\ e_k \\
        % or an abstraction,
        | &\ \lambda x.\ e \\
        % or an application,
        | &\ e\ e \\
        % or a fixpoint,
        | &\ \fix{ f }{ e }
      \end{align*} }
    %
    \ffigbox[.48\textwidth] %
    {\caption{The space of types, $\types$.
        %
        Metavariable $\tau$ ranges over $\types$, %
        $x_i$ and $y_i$ range over individual value variables, %
        $\overline{x}$ range over vectors of value variables, and %
        $X$ ranges over type variables.} %
      \label{fig:types} }
    %
    { \begin{align*} %
        % a type is a base type,
        \tau \rightarrow &\ \setformer{ \nu : \bools }{ \varphi } \\
        | &\ \setformer{ \nu : \ints}{ \varphi } \\
        % a type variable,
        | &\ X \overline{x} \\
        % a function type, or
        | &\ x : \tau \rightarrow \tau \\
        % a recursive type
        | &\ \mu X \overline{x}.\ %
        C_0(y_0^0 : \tau_0^0, \ldots, %
        y_0^{ \arityof{C_0} } : \tau_0^{ \arityof{C_0} }), %
        \ldots, \\
        &\ C_n(y_n^0 : \tau_n^0, \ldots, %
        y_n^{ \arityof{C_n} } : \tau_n^{ \arityof{C_n} })
      \end{align*} }
  \end{floatrow}
\end{figure*}

% define STLC syntax:
\paragraph{Syntax}
% define spaces:
\autoref{fig:exprs} defines the space of
simply-typed $\lambda$-calculus expressions, $\expr$.
%
For the remainder of this paper, metavariable $e$ ranges over
$\expr$, %
$c$ ranges over constants, %
$\oplus$ ranges over binary operators, $x$ and $f$ range over
program variables, and $C$ ranges over constructor symbols.
%
Each constructor symbol $C$ is mapped to a fixed cardinality, denoted
$|C| \in \nats$.

% define expressions:
An expression is a Boolean or integer constant, %
a variable (drawn from space of variables $\vars$), %
an application of an fixed operator to a pair of expressions, %
an application of a constructor to a sequence of expressions, %
a match expression, %
a $\lambda$-abstraction, %
an application, or %
a fixpoint expression.
%
For each $e \in \expr$, the \emph{free-variables} of $e$ are denoted
$\freevars(e)$.
%
A \emph{refinement-typed $\lambda$-calculus (RTLC)} \emph{program} is
a closed expression---i.e., an expression $e$ such that $\freevars(e)
= \emptyset$.
%
The space of RTLC programs is denoted $\rtlc$.
%
For $P \in \rtlc$, the set of all subexpressions of $P$ is denoted
$\subexps(P)$.

% define semantics:
\paragraph{Semantics}
%
\begin{figure*}
  \centering
  \begin{gather*}
  % evaluate const:
  \inference[ CONST ]{ }{ \Gamma \judges c \evalsto c } 
  % evaluate variable:
  \inference[ VAR ]{ }{ \Gamma \judges x \evalsto \Gamma(x) }
  % evaluate binary operations:
  \inference[ OP ]{ %
    \Gamma \judges e_0 \evalsto v_0 & %
    \Gamma \judges e_1 \evalsto v_1 } %
  { \Gamma \judges e_0 \oplus e_1 \evalsto v_0 \oplus v_1 } \\ \\
  % evaluate constructor application:
  \inference[ CONS ]{ %
    \Gamma \judges e_0 \evalsto v_0 & \ldots & %
    \Gamma \judges e_k \evalsto v_k } %
  { \Gamma \judges C(e_0, \ldots, e_k) \evalsto C(v_0, \ldots, v_k) }
  % evaluate if-then-else by then branch
  \inference[ $\mathrm{MATCH}_{i}$ ]{ %
    \Gamma \judges e \evalsto C_i(v_0, \ldots, v_{ |C_i| }) & %
    X_i = x_1, \ldots, x_{|C_i|} \\ %
    \Gamma[ x_0 \mapsto v_0, \ldots, x_{|C_i|} \mapsto v_{|C_i|} ]
    \judges %
    e_i \evalsto v' } %
  { \Gamma \judges \cc{match}\ e\ \cc{with}\ %
    C_0(X_0)\ \cc{->}\ e_0 %
    \ldots %
    C_k(X_k)\ \cc{->}\ e_k \evalsto v' } \\ \\
  % evaluate an abstraction:
  \inference[ ABS ]{ } %
  { \Gamma \judges \lambda x.\ e \evalsto \lambda x.\ e }
  % apply a lambda:
  \inference[ APP-ABS ]{ %
    \Gamma \judges e_0 \evalsto \lambda x.\ e_0' & %
    \Gamma \judges e_1 \evalsto v_1 & %
    \upd{\Gamma}{x}{v_1} \judges e_0' \evalsto v }
  { \Gamma \judges e_0\ e_1 \evalsto v } \\ \\
  % evaluate a fix:
  \inference[ FIX ]{ } %
  { \Gamma \judges \fix{ f }{ e } \evalsto \fix{ f }{ e } } 
  % apply a fixpoint:
  \inference[ APP-FIX ]{ %
    \Gamma \judges e_0 \evalsto \fix{ f }{ e_0' } & %
    \upd{ \Gamma }{ f }{ \fix{ f }{ e_0' } } \judges %
    e_0'\ e_1 \evalsto v }
  { \Gamma \judges e_0\ e_1 \evalsto v } 
  \end{gather*}
  \caption{The natural semantics of $\rtlc$.}
  \label{fig:nat-sem}
\end{figure*}
%
A value is a Boolean, an integer, a $\lambda$ abstraction, a fixpoint
abstraction, or a constructor applied to a sequence of values;
%
the space of values is denoted $\values$.
%
The semantics of $\rtlc$ is defined by a natural semantics that
relates expressions to values, given in \autoref{fig:nat-sem}.
%
For $e \in \expr$ and $v \in \values$, the fact that $e$
\emph{evaluates} to $v$ is denoted $e \evalsto v$.

\paragraph{Typing}
%
Each expression has types that describe the spaces of values that it
takes as input and the spaces of values to which it evaluates.
%
The space of types, defined over a space of type variable $\typevars$
is given in \autoref{fig:types}.
%
A type is either a refined Boolean value, %
a refined Integer value, %
a type vector applied to a vector of value variables, %
%
\BH{does this also need to have a refinement?}
%
a function over types, %
%
or a recursive type, defined over a type parameter $X$ abstracted over
a sequence of value variables $\overline{x}$.
%
For each $\tau \in \types$, the free value variables and free type
variables of $\tau$, denoted $\freevaluevars(\tau) \subseteq \vars$
and $\freetypevars(\tau) \subseteq \typevars$, are the value variables
and type variables that occur in $\tau$ that are not bound in a
recursive type definition.
%
$\tau$ is \emph{well-formed} if $\freevars(\tau) = \emptyset$.
%
Note that $\tau$ may contain occurrences of free value variables.

% introduce has-type symbol:
For expression $e \in \expr$ and type $\tau \in \types$, the fact that
$e$ has type $\tau$ is denoted $e \hastype \tau$.
%
The definition of the has-type relation is standard.

% define base types:
A type $\tau$ in which the only formula is that occurs in a component
type of $\tau$ is $\true$ is a \emph{base} type, denoted
$\basetype(\tau)$.
%
Each expression that has a type has a unique base type, which can be
inferred automatically by a conventional type-inference algorithm,
such as Hindley-Milner algorithm.
%
\sys assumes access to an algorithm that, given $e \in \expr$,
computes $\basetype(\tau)$.
%
\BH{extend this to give base type of all free variables of an
  expression}

% problem statement
The problem that we address in this paper is, given program $P \in
\rtlc$ and well-formed $\tau \in \types$, to determine if $P \hastype
\tau$.

\paragraph{Symbolic encoding of semantics}
%
The semantics of $\rtlc$ is represented symbolically using \lia
formulas.
%
\BH{hoist defn of disjoint copies of variables for expressions}
%
In particular, for each $e \in \expr$ with subexpressions $e_0,
\ldots, e_n \in \expr$, there is some $\varphi \in \formulas{\vars[e],
  \vars[e_0], \ldots, \vars[e_n]}$ such that

\BH{merge updated defn from current draft of equivalence paper}

% define CHC's:
\subsection{Constrained Horn Clauses}
\label{sec:chcs}

\subsubsection{Structure}
% definition of CHC
A Constrained Horn Clause is a body, consisting of a conjunctive set
of uninterpreted relational predicates and a constraint, and a head
relational predicate.
%
Relational predicates are predicate symbols paired with a map from
each symbol to its arity.
%
\begin{defn}
  \label{defn:rel-preds}
  For each space of symbols $\mathcal{R}$ and function $a: \mathcal{R}
  \to \nats$, $(\mathcal{R}, a)$ is a space of \emph{relational
    predicates}.
\end{defn}
%
The space of all relational predicates is denoted $\relpreds$.
%
For each space of relational predicates $\mathcal{R} \in \relpreds$,
we denote the predicate symbols and arity of $\mathcal{R}$ as
$\relsof{\mathcal{R}}$ and $\arityof{\mathcal{R}}$, respectively.
%
For relational predicates $\mathcal{R} \in \relpreds$ and symbol $R$,
we denote $R \in \relsof{ \mathcal{R} }$ alternatively as $R \in
\mathcal{R}$.
%
All definitions introduced in this section are given over a fixed,
arbitrary set of relational-predicate symbols $\mathcal{R}$.

% define applications of relational predicates:
An application of a relational predicate is a relational-predicate
symbol $R$ paired with a sequence of variables of length equal to the
arity of $R$.
%
\begin{defn}
  \label{defn:pred-apps}
  For $R \in \mathcal{R}$ and sequence of variables $Y \in X^{*}$ such
  that $|Y| = \arityof{\mathcal{R}}(R)$, $(R, Y)$ is an
  \emph{application}.
\end{defn}
%
The space of applications is denoted $\apps{\mathcal{R}}$.
%
For each application $A \in \apps{\mathcal{R}}$, the predicate symbol
and argument sequence of $A$ are denoted $\relof{A}$ and $\argsof{A}$
respectively.

% define CHC:
A Constrained Horn Clause is a body of applications, a constraint, and
a head application.
%
\begin{defn}
  \label{defn:chcs-structure}
  For $\mathcal{A} \subseteq \apps{\mathcal{R}}$ and %
  $\varphi \in \formulas$, $\mathcal{B} = (\mathcal{A}, \varphi)$ is a
  \emph{clause body}.
  % 
  For $H \in \apps{\mathcal{R}}$, $(\mathcal{B}, H)$ is a
  \emph{Constrained Horn Clause}.
\end{defn}
% define space of bodies:
The space of clause bodies is denoted $\bodies{ \mathcal{R} } =
\pset(\apps{\mathcal{R}}) \times \formulas$.
% define space, projection:
The space of Constrained Horn Clauses is denoted $\chc{\mathcal{R}} =
\bodies{ \mathcal{R} } \times \apps{\mathcal{R}}$.
%
For each clause $\mathcal{C} \in \chc{\mathcal{R}}$, the body and head
of $\mathcal{C}$ are denoted $\bodyof{ \mathcal{C} }$ and $\headof{
  \mathcal{C} }$, respectively.
% define siblings:
For each $\mathcal{S} \subseteq \chc{ \mathcal{R} }$, $\mathcal{C} \in
\mathcal{S}$, and all applications $A_0, A_1 \in \appsof{ \bodyof{
    \mathcal{C} } }$, $\relof{A_0}$ and $\relof{A_1}$ are
\emph{siblings} in $\mathcal{S}$.

% define system of CHC's
We will present \sys as a solver for CHCs without recursive
definitions of predicates, represented in a normalized form.
% give conditions on normalization:
Let $\mathcal{S} \subseteq \chc{ \mathcal{R} }$ be such that %
\textbf{(1)} each relational predicate occurs in the head of some
clause, %
\textbf{(2)} each $R \in \mathcal{R}$ is applied to the same variables
$\varsof{ \mathcal{S}}(R)$ in each of its occurrences in a clause as a
head, %
\textbf{(3)} all clauses with distinct head relational predicates are
defined over disjoint spaces of variables, %
\textbf{(4)} each relational predicate occurs in each clause body in
at most one application, and %
\textbf{(5)} there is exactly one relational predicate $\queryof{
  \mathcal{S} }$ that is the dependency of no relational predicate.
%
Then $\mathcal{S}$ is a \emph{normalized} recursion-free system.
%
For the remainder of this paper, we consider only normalized sets of
CHCs, and denote the space of such sets as $\chcs{\mathcal{R}}$.

\subsubsection{Solutions}
\label{sec:chc-solns}
%
A solution to a CHC system $\mathcal{S}$ is an interpretation of each
relational predicate $R$ of arity $n$ as a formula over $n$ free
variables such that for each clause $\mathcal{C} \in \mathcal{S}$, the
conjunction of interpretations of all relational predicates in the
body of $\mathcal{C}$ and the constraint of $\mathcal{C}$ entail the
interpretation of the head of $\mathcal{C}$.
%
Let a map from each $R \in \relsof{ \mathcal{R} }$ to a formula over
an ordered vector of $\arityof{ \mathcal{R} }(R)$ free variables be an
\emph{interpretation} of $\mathcal{R}$;
%
let the space of interpretations of $\mathcal{R}$ be denoted
$\interps{ \mathcal{R} }$.
%
\begin{defn}
  \label{defn:chc-soln}
  For $\mathcal{B} \in \bodies{ \mathcal{R} }$ and $H \in \apps{
    \mathcal{R} }$, %
  let $\sigma \in \interps{ \mathcal{R} }$ be such that for each $R
  \in \mathcal{R}$, $\arityof{\mathcal{R}}(R) = \degreeof{\sigma(R)}$
  and %
  %
  \[ \elts{ \sigma(\relof{ A })[ \argsof{ A } ] }_{ A \in \appsof{
      \mathcal{B} } }, %
  \ctrof{ \mathcal{B} } \entails %
  \sigma( \relof{H} )[ \argsof{ H } ]
  \]
  Then $\sigma$ is a \emph{solution} of $(\mathcal{B}, H)$.
\end{defn}
% define partial solution 
For $\mathcal{S} \subseteq \chc{ \mathcal{R} }$, if
% solution respects dependence order:
\textbf{(a)} for each $R \in \domain(\sigma)$ and $R' \in \mathcal{R}$
a dependence of $R$ in $\mathcal{S}$, it holds that $R' \in
\domain(\sigma)$;
%
\textbf{(b)} for each $\mathcal{C} \in \mathcal{S}$ such that
$\relof{ \headof{ \mathcal{C} } } \in \domain(\sigma)$, $\sigma$ is a
solution of $\mathcal{C}$;
%
then $\sigma$ is a \emph{partial solution} of $\mathcal{S}$.
%
If, in addition, $\sigma(\queryof{ \mathcal{S} }) \entails \false$,
then $\sigma$ is a \emph{solution} of $\mathcal{S}$.
%


%%% Local Variables: 
%%% mode: latex
%%% TeX-master: "p"
%%% End: 
